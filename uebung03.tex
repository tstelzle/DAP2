\documentclass{article}
\usepackage{amsmath, amssymb}
\title{DAP2 Uebung Blatt 3\\Gruppe 9}
\author{Tarek Stelzle 190736, Frederic Arnold 183831}
\begin{document}
\maketitle

\section*{Aufgabe 3.1}

\subsection*{a)}
Das Programm "BerechneWert1" berechnet iterative die n-te Fibonaccizahl.

\subsection*{b)}
Zu Beginn des Schleifendurchlaufs mit i = k gilt:\\
B(k)= fib(k)\\
A(k)= fib(k-1) + k\\
\\
Induktionsanfang:\\
B(1)= 1 =fib(1)\\
A(k)= 1 =fib(0) + 1 = 0+1 =1\\
\\
Induktionsvorrausetzung:\\
Die Aussage gelte fuer ein beliebiges und festes k $\in$ $\mathbb{N}$.\\
\\
Induktionsschritt:\\
B(k+1)\\
= c + A(k)\\
= B(k) - k + A(k)\\
= fib(k) - k + A(k)\\
= fib(k) - k + fib(k-1) + k\\
= fib(k) + fib(k-1)\\
= fib(k+1)\\
\\
A(k+1)\\
= c + 2k + 1\\
= B(k) - k + 2k +1\\
= fib(k) - k + 2k + 1\\
= fib(k) + k + 1\\
= fib(k + 1 - 1) + (k+1)\\
\\
\subsection*{c)}
In der Ausfuehrung des For-Schleifen-Koerpers in der i=k, wird b gesetzt auf B(k+1).\\
Da der Koerper der For-Schleife beim letzten Mal mit i=n-1 aufgerufen wird, gibt das Programm den Wert B(n-1+1) = fib(n) aus.\\
\\
\section*{Aufgabe 3.2)}
Behauptung:\\
Das Programm "BerechneWert2" berechnet rekursiv die n-te Fibonaccizahl.\\
\\
Induktionsanfang:\\
BerechneWert2(0) = 0\\
BerechneWert2(1) = 1\\
BerechneWert2(2) = 2\\
\\
Induktionsvorrausetzung:\\
Fuer ein beliebiges, festes n $\in$ $\mathbb{N}$ gelte die Aussage fuer jedes k $\in$ $\mathbb{N}$, mit k $\le$ n.\\
\\
Induktionsschritt:\\
BerechneWert2(n+1) \\
= 2 * BerechneWert2(n+1-1) - BerechneWert2(n+1-3)\\
= 2 * BerechneWert2(n) - BerechneWert2(n-2)\\
= 2 * fib(n) - fib(n-2)\\
= fib(n) + fib(n-1) + fib(n-2) - fib(n-2)\\
= fib(n) + fib(n-1)\\
= fib(n+1)\\ 

\end{document}
